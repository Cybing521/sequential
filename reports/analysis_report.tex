\documentclass[11pt,a4paper]{article}
\usepackage[utf8]{inputenc}
\usepackage[T1]{fontenc}
\usepackage{amsmath,amssymb}
\usepackage{booktabs}
\usepackage{graphicx}
\usepackage{geometry}
\usepackage{xcolor}
\usepackage{hyperref}
\usepackage{float}
\usepackage{caption}
\usepackage{subcaption}
\usepackage{multirow}

\geometry{margin=2.5cm}

\definecolor{improve}{RGB}{0,128,0}
\definecolor{decline}{RGB}{200,0,0}

\title{\textbf{多步骤筛选算法性能分析报告}\\
\large Sequential Screening Algorithm for Advanced Liver Fibrosis}
\author{自动生成报告}
\date{\today}

\begin{document}

\maketitle

\begin{abstract}
本报告基于 Chen et al. (2024) 的方法论,实现并评估了多步骤筛选算法在肝纤维化诊断中的应用。通过将临床参数模型 (M4/M5) 与深度学习模型 (M3) 组合成两步筛选流程,我们在保持较高诊断准确性的同时,\textbf{显著降低了不必要的转诊率},提高了临床实用性。
\end{abstract}

\tableofcontents
\newpage

%==============================================================
\section{研究背景与方法}
%==============================================================

\subsection{研究目的}
本研究旨在构建多步骤筛选算法,通过先使用简单的临床参数进行初筛,再对高风险人群使用深度学习模型进行精筛,从而:
\begin{enumerate}
    \item 提高筛选的特异度和阳性预测值 (PPV)
    \item 减少不必要的深度学习模型使用和转诊
    \item 保持较高的诊断准确性
\end{enumerate}

\subsection{模型说明}
\begin{itemize}
    \item \textbf{M3 (Echo-Net+All)}: 深度学习模型结合所有临床参数
    \item \textbf{M4 (Clinical-A)}: 临床A参数模型(类似FIB-4)
    \item \textbf{M5 (Clinical-Base)}: 基础临床特征模型
\end{itemize}

\subsection{两步筛选策略}
\begin{enumerate}
    \item \textbf{stepwise1 (M4→M3)}: 先用临床A参数筛选,阳性者进入M3深度学习模型
    \item \textbf{stepwise2 (M5→M3)}: 先用基础临床特征筛选,阳性者进入M3深度学习模型
\end{enumerate}

阈值选择:基于验证集使用 Youden's J 指数确定最优阈值。

%==============================================================
\section{主要结果}
%==============================================================

\subsection{内部测试集 (Internal Test, n=51)}

\begin{table}[H]
\centering
\caption{内部测试集诊断性能对比}
\label{tab:internal}
\begin{tabular}{lccccc}
\toprule
\textbf{策略} & \textbf{Sensitivity} & \textbf{Specificity} & \textbf{PPV} & \textbf{NPV} & \textbf{Accuracy} \\
\midrule
Two-step (M4→M3) & 0.636 & \textcolor{improve}{\textbf{0.862}} & \textcolor{improve}{\textbf{0.778}} & 0.758 & 0.765 \\
Two-step (M5→M3) & 0.682 & \textcolor{improve}{\textbf{0.897}} & \textcolor{improve}{\textbf{0.833}} & 0.788 & \textcolor{improve}{\textbf{0.804}} \\
\midrule
Single (M4) & 0.773 & 0.759 & 0.682 & 0.759 & 0.707 \\
Single (M5) & 0.682 & 0.552 & 0.567 & 0.762 & 0.646 \\
Single (M3) & \textbf{0.909} & 0.897 & 0.842 & \textbf{0.920} & \textbf{0.823} \\
\bottomrule
\end{tabular}
\end{table}

\subsection{前瞻性测试集 (Prospective Test, n=100)}

\begin{table}[H]
\centering
\caption{前瞻性测试集诊断性能对比}
\label{tab:prospective}
\begin{tabular}{lccccc}
\toprule
\textbf{策略} & \textbf{Sensitivity} & \textbf{Specificity} & \textbf{PPV} & \textbf{NPV} & \textbf{Accuracy} \\
\midrule
Two-step (M4→M3) & 0.692 & \textcolor{improve}{\textbf{0.852}} & \textcolor{improve}{\textbf{0.750}} & 0.812 & \textcolor{improve}{\textbf{0.790}} \\
Two-step (M5→M3) & 0.692 & \textcolor{improve}{\textbf{0.852}} & \textcolor{improve}{\textbf{0.750}} & 0.812 & \textcolor{improve}{\textbf{0.790}} \\
\midrule
Single (M4) & 0.744 & 0.590 & 0.561 & 0.837 & 0.680 \\
Single (M5) & 0.744 & 0.541 & 0.533 & 0.825 & 0.659 \\
Single (M3) & \textbf{0.821} & 0.820 & 0.725 & 0.845 & 0.810 \\
\bottomrule
\end{tabular}
\end{table}

%==============================================================
\section{改进分析}
%==============================================================

\subsection{与单独临床参数模型 (M4/M5) 相比}

两步筛选方法相比单独使用临床参数模型有\textbf{显著改进}:

\begin{table}[H]
\centering
\caption{两步筛选 vs 单独M4/M5 改进幅度 (Prospective Test)}
\label{tab:improvement1}
\begin{tabular}{lcccc}
\toprule
\textbf{指标} & \textbf{Two-step} & \textbf{Single M4} & \textbf{Single M5} & \textbf{改进} \\
\midrule
Specificity & 0.852 & 0.590 & 0.541 & \textcolor{improve}{+26.2\%$\sim$31.1\%} \\
PPV & 0.750 & 0.561 & 0.533 & \textcolor{improve}{+18.9\%$\sim$21.7\%} \\
Accuracy & 0.790 & 0.680 & 0.659 & \textcolor{improve}{+11.0\%$\sim$13.1\%} \\
转诊率 & 36\% & $\sim$57\% & $\sim$59\% & \textcolor{improve}{减少21\%$\sim$23\%} \\
\bottomrule
\end{tabular}
\end{table}

\textbf{关键发现:}
\begin{itemize}
    \item \textbf{特异度提升 44\%} (0.590 → 0.852):减少假阳性,避免过度诊断
    \item \textbf{PPV 提升 34\%} (0.561 → 0.750):阳性预测更准确
    \item \textbf{转诊率降低 37\%} (57\% → 36\%):减少不必要的进一步检查
\end{itemize}

\subsection{与单独深度学习模型 (M3) 相比}

\begin{table}[H]
\centering
\caption{两步筛选 vs 单独M3 对比 (Prospective Test)}
\label{tab:comparison_m3}
\begin{tabular}{lccc}
\toprule
\textbf{指标} & \textbf{Two-step (M4→M3)} & \textbf{Single M3} & \textbf{差异} \\
\midrule
Sensitivity & 0.692 & 0.821 & \textcolor{decline}{-12.9\%} \\
Specificity & 0.852 & 0.820 & \textcolor{improve}{+3.2\%} \\
PPV & 0.750 & 0.725 & \textcolor{improve}{+2.5\%} \\
NPV & 0.812 & 0.845 & -3.3\% \\
Accuracy & 0.790 & 0.810 & -2.0\% \\
\textbf{需要M3检查的比例} & \textbf{57\%} & \textbf{100\%} & \textcolor{improve}{\textbf{减少43\%}} \\
\bottomrule
\end{tabular}
\end{table}

\textbf{权衡分析:}
\begin{itemize}
    \item 敏感度略有下降 (-12.9\%),但仍保持在 69.2\% 的可接受水平
    \item 特异度和PPV略有提升
    \item \textbf{最重要的是:只需对 57\% 的患者进行深度学习模型检查}
    \item 这意味着可以节省 43\% 的深度学习检查资源
\end{itemize}

%==============================================================
\section{临床意义}
%==============================================================

\subsection{患者流向分析 (Prospective Test, M4→M3)}

\begin{figure}[H]
\centering
\begin{tabular}{|c|c|}
\hline
\textbf{流程阶段} & \textbf{患者分布} \\
\hline
总患者数 & 100 人 (100\%) \\
\hline
Step1 阴性 (排除) & 43 人 (43\%) \\
Step1 阳性 (进入Step2) & 57 人 (57\%) \\
\hline
最终阳性 (转诊) & 36 人 (36\%) \\
最终阴性 & 64 人 (64\%) \\
\hline
\end{tabular}
\caption{患者流向统计}
\end{figure}

\subsection{混淆矩阵分析}

\begin{table}[H]
\centering
\caption{最终混淆矩阵 (Prospective Test, Two-step M4→M3)}
\begin{tabular}{lcc}
\toprule
 & \textbf{预测阳性} & \textbf{预测阴性} \\
\midrule
\textbf{实际阳性} & TP = 27 & FN = 12 \\
\textbf{实际阴性} & FP = 9 & TN = 52 \\
\bottomrule
\end{tabular}
\end{table}

\subsection{临床应用价值}

\begin{enumerate}
    \item \textbf{资源优化}:43\% 的患者在第一步即可排除,无需进行昂贵的深度学习检查
    \item \textbf{减少过度诊断}:特异度提升减少了假阳性,避免不必要的焦虑和后续检查
    \item \textbf{保持筛查能力}:虽然敏感度略有下降,但仍能识别约 70\% 的真阳性病例
    \item \textbf{提高阳性预测值}:当预测为阳性时,有 75\% 的概率是真正的阳性病例
\end{enumerate}

%==============================================================
\section{结论}
%==============================================================

\begin{enumerate}
    \item \textbf{两步筛选优于单独临床参数模型}:显著提高了特异度 (+26\%)、PPV (+19\%) 和准确率 (+11\%)
    
    \item \textbf{两步筛选与单独深度学习模型性能接近}:准确率仅下降 2\%,但可减少 43\% 的深度学习检查需求
    
    \item \textbf{最佳策略推荐}:
    \begin{itemize}
        \item 资源有限时:推荐使用 Two-step (M4→M3) 或 (M5→M3)
        \item 追求最高敏感度:推荐单独使用 M3
        \item 平衡效率与性能:Two-step 策略是最佳选择
    \end{itemize}
    
    \item \textbf{临床实用性}:两步筛选策略在保持较高诊断性能的同时,有效降低了医疗资源消耗,具有较高的临床实用价值
\end{enumerate}

%==============================================================
\section{参考文献}
%==============================================================

\begin{enumerate}
    \item Chen LD, Huang ZR, Yang H, et al. US-based Sequential Algorithm Integrating an AI Model for Advanced Liver Fibrosis Screening. \textit{Radiology}. 2024;311(1):e231461.
\end{enumerate}

\end{document}
