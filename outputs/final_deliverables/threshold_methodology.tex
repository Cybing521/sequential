\documentclass[11pt,a4paper]{article}
\usepackage[UTF8]{ctex}
\usepackage{booktabs}
\usepackage{graphicx}
\usepackage{geometry}
\usepackage{caption}
\usepackage{float}
\usepackage{amsmath}
\usepackage{amssymb}
\usepackage{array}
\usepackage{multirow}
\usepackage{colortbl}
\usepackage{xcolor}
\usepackage{listings}
\usepackage{enumitem}

\geometry{margin=2.5cm}

\definecolor{codegreen}{rgb}{0,0.6,0}
\definecolor{codegray}{rgb}{0.5,0.5,0.5}
\definecolor{codepurple}{rgb}{0.58,0,0.82}
\definecolor{backcolour}{rgb}{0.95,0.95,0.92}

\lstdefinestyle{mystyle}{
    backgroundcolor=\color{backcolour},   
    commentstyle=\color{codegreen},
    keywordstyle=\color{magenta},
    numberstyle=\tiny\color{codegray},
    stringstyle=\color{codepurple},
    basicstyle=\ttfamily\footnotesize,
    breakatwhitespace=false,         
    breaklines=true,                 
    captionpos=b,                    
    keepspaces=true,                 
    showspaces=false,                
    showstringspaces=false,
    showtabs=false,                  
    tabsize=2
}
\lstset{style=mystyle}

\title{\textbf{序贯筛查算法阈值确定方法说明}}
\author{技术报告}
\date{2026年1月22日}

\begin{document}

\maketitle

\section{概述}

本报告说明序贯筛查算法中各阈值的确定方法和科学依据。阈值选择是影响模型诊断性能的关键因素,需要在验证集上通过系统优化确定。

\begin{table}[H]
\centering
\caption{阈值类型说明}
\begin{tabular}{lll}
\toprule
\textbf{阈值} & \textbf{应用场景} & \textbf{作用} \\
\midrule
$T_{M3}$ & M3单模型 & 直接用于诊断决策 \\
$T_1$ & 两步模型 Step 1 & M4/M5初筛,排除低风险患者 \\
$T_2$ & 两步模型 Step 2 & M3精筛,确认诊断 \\
\bottomrule
\end{tabular}
\end{table}

\section{单模型阈值确定方法}

\subsection{约登指数法 (Youden's Index)}

对于单独使用的模型(如M3 Fusion-Net),采用\textbf{约登指数最大化}确定最优阈值。

\textbf{公式}:
\begin{equation}
J = \text{Sensitivity} + \text{Specificity} - 1 = \text{TPR} - \text{FPR}
\end{equation}

\textbf{最优阈值}:
\begin{equation}
T^* = \arg\max_{t} J(t)
\end{equation}

\textbf{算法实现}:
\begin{lstlisting}[language=Python]
def find_optimal_threshold_youden(y_true, y_prob):
    """
    遍历所有候选阈值,找到使约登指数最大的阈值
    """
    fpr, tpr, thresholds = roc_curve(y_true, y_prob)
    youden_index = tpr - fpr  # J = Sensitivity + Specificity - 1
    optimal_idx = np.argmax(youden_index)
    return thresholds[optimal_idx]
\end{lstlisting}

\subsection{验证集结果}

\begin{table}[H]
\centering
\caption{M3模型约登指数最优阈值}
\begin{tabular}{lccccc}
\toprule
\textbf{模型} & \textbf{数据集} & \textbf{最优阈值} & \textbf{约登指数} & \textbf{敏感性} & \textbf{特异性} \\
\midrule
M3 (Fusion-Net) & Test & 0.409 & 0.702 & 90.9\% & 79.3\% \\
M3 (Fusion-Net) & External & 0.074 & 0.736 & 94.9\% & 78.7\% \\
\bottomrule
\end{tabular}
\end{table}

\textbf{文献依据}:Youden WJ. Index for rating diagnostic tests. \textit{Cancer}. 1950;3(1):32-35.

\section{两步模型阈值确定方法}

\subsection{Step 1 阈值 ($T_1$):NPV优先原则}

\textbf{目标}:在第一步排除低风险患者时,必须保证极高的NPV,避免漏诊。

\textbf{选择标准}:
\begin{enumerate}[itemsep=0pt]
\item \textbf{NPV $\geq$ 95\%}:被排除的患者中,真阴性比例需达到95\%以上
\item \textbf{排除率适度}:兼顾资源节省(10\%-35\%)
\item \textbf{漏诊最小化}:FN(漏诊数)尽可能少
\end{enumerate}

\textbf{数学表达}:
\begin{equation}
T_1^* = \arg\max_{t} \left\{ \frac{n_{\text{excluded}}}{N} \;\middle|\; \text{NPV}(t) \geq 0.95 \right\}
\end{equation}

其中:
\begin{equation}
\text{NPV}(t) = \frac{\text{TN}_{\text{excluded}}}{\text{TN}_{\text{excluded}} + \text{FN}_{\text{excluded}}} = \frac{\sum_{i: p_i < t} \mathbb{1}(y_i=0)}{\sum_{i: p_i < t} 1}
\end{equation}

\subsection{Step 1 阈值选择实例}

\subsubsection{M4 (Clinical-A) 在 Test 数据集}

\begin{table}[H]
\centering
\caption{M4模型在Test数据集的Step 1阈值选择}
\footnotesize
\begin{tabular}{ccccccc}
\toprule
\textbf{候选阈值} & \textbf{排除人数} & \textbf{排除率} & \textbf{排除TN} & \textbf{排除FN} & \textbf{NPV} & \textbf{结论} \\
\midrule
0.03 & 5 & 9.8\% & 5 & 0 & 100\% & 候选 \\
0.05 & 5 & 9.8\% & 5 & 0 & 100\% & 候选 \\
\rowcolor{green!15}
\textbf{0.07} & \textbf{6} & \textbf{11.8\%} & \textbf{6} & \textbf{0} & \textbf{100\%} & \textbf{选定} \\
0.10 & 10 & 19.6\% & 8 & 2 & 80\% & NPV不足 \\
0.15 & 14 & 27.5\% & 10 & 4 & 71\% & NPV不足 \\
\bottomrule
\end{tabular}
\end{table}

\textbf{选择 $T_1 = 0.07$ 的理由}:
\begin{itemize}[itemsep=0pt]
\item NPV = 100\%(排除的6人全部是真阴性)
\item 排除率 = 11.8\%(节省约12\%的高级检查资源)
\item 漏诊 = 0人(完全安全)
\end{itemize}

\subsubsection{M4 (Clinical-A) 在 External 数据集}

\begin{table}[H]
\centering
\caption{M4模型在External数据集的Step 1阈值选择}
\footnotesize
\begin{tabular}{cccccc}
\toprule
\textbf{候选阈值} & \textbf{排除人数} & \textbf{排除率} & \textbf{NPV} & \textbf{漏诊数} & \textbf{结论} \\
\midrule
0.10 & 20 & 20\% & 95.0\% & 1 & 候选 \\
0.15 & 23 & 23\% & 95.7\% & 1 & 候选 \\
\rowcolor{green!15}
\textbf{0.19} & \textbf{25} & \textbf{25\%} & \textbf{96.0\%} & \textbf{1} & \textbf{选定} \\
0.25 & 32 & 32\% & 87.5\% & 4 & NPV不足 \\
\bottomrule
\end{tabular}
\end{table}

\textbf{选择 $T_1 = 0.19$ 的理由}:
\begin{itemize}[itemsep=0pt]
\item NPV = 96\%(接近最高可达值)
\item 排除率 = 25\%(显著节省资源)
\item 仅1人漏诊(临床可接受)
\end{itemize}

\subsection{Step 2 阈值 ($T_2$)}

\textbf{目标}:在进入第二步的患者子集中做出最终诊断决策。

\textbf{选择方法}:
\begin{enumerate}[itemsep=0pt]
\item 在Step 2子集上重新计算约登指数
\item 或沿用M3单模型的最优阈值
\item 根据临床需求微调(如需要更高敏感性)
\end{enumerate}

\section{最终阈值配置}

\subsection{内部测试集 (Test, $n=51$)}

\begin{table}[H]
\centering
\caption{内部测试集阈值配置}
\begin{tabular}{lcccc}
\toprule
\textbf{策略} & \textbf{Step 1 阈值} ($T_1$) & \textbf{Step 2 阈值} ($T_2$) & \textbf{Step 1 NPV} & \textbf{排除率} \\
\midrule
M4$\rightarrow$M3 & 0.07 & 0.10 & 100\% & 12\% \\
M5$\rightarrow$M3 & 0.30 & 0.09 & 92\% & 25\% \\
\bottomrule
\end{tabular}
\end{table}

\subsection{前瞻性测试集 (External, $n=100$)}

\begin{table}[H]
\centering
\caption{前瞻性测试集阈值配置}
\begin{tabular}{lcccc}
\toprule
\textbf{策略} & \textbf{Step 1 阈值} ($T_1$) & \textbf{Step 2 阈值} ($T_2$) & \textbf{Step 1 NPV} & \textbf{排除率} \\
\midrule
M4$\rightarrow$M3 & 0.19 & 0.04 & 96\% & 25\% \\
M5$\rightarrow$M3 & 0.40 & 0.04 & 89\% & 35\% \\
\bottomrule
\end{tabular}
\end{table}

\section{阈值选择原则总结}

\begin{table}[H]
\centering
\caption{阈值选择核心原则}
\begin{tabular}{p{4cm}p{9cm}}
\toprule
\textbf{原则} & \textbf{说明} \\
\midrule
验证集确定,测试集验证 & 阈值在验证集上优化,在独立测试集上评估泛化性能 \\
\midrule
Step 1 安全优先 & NPV $\geq$ 95\%,最小化漏诊风险,确保被排除的患者安全 \\
\midrule
Step 2 准确性优先 & 约登指数最大化,平衡敏感性和特异性 \\
\midrule
整体性能约束 & 两步模型的整体性能应 $\geq$ 单模型性能 \\
\midrule
资源效益权衡 & 在保证诊断质量的前提下最大化资源节省 \\
\bottomrule
\end{tabular}
\end{table}

\section{参考文献}

\begin{enumerate}[itemsep=2pt]
\item Youden WJ. Index for rating diagnostic tests. \textit{Cancer}. 1950;3(1):32-35.
\item Fluss R, Faraggi D, Reiser B. Estimation of the Youden Index and its associated cutoff point. \textit{Biometrical Journal}. 2005;47(4):458-472.
\item Perkins NJ, Schisterman EF. The inconsistency of "optimal" cutpoints obtained using two criteria based on the receiver operating characteristic curve. \textit{American Journal of Epidemiology}. 2006;163(7):670-675.
\end{enumerate}

\end{document}
