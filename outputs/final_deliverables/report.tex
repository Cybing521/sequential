\documentclass[11pt,a4paper]{article}
\usepackage[UTF8]{ctex}
\usepackage{booktabs}
\usepackage{graphicx}
\usepackage{geometry}
\usepackage{caption}
\usepackage{float}
\usepackage{amsmath}
\usepackage{threeparttable}
\usepackage{array}
\usepackage{multirow}
\usepackage{colortbl}
\usepackage{xcolor}

\geometry{margin=2cm}

\title{晚期肝纤维化序贯筛查算法\\
\large 性能分析报告}
\author{自动生成分析}
\date{\today}

\begin{document}

\maketitle

\section{引言}

本报告呈现了两种两步序贯筛查算法的性能比较分析:
\begin{itemize}
\item \textbf{M4$\rightarrow$M3}: Clinical-A(临床参数)$\rightarrow$ Fusion-Net(最优配置)
\item \textbf{M5$\rightarrow$M3}: Clinical-Net(基础临床深度学习)$\rightarrow$ Fusion-Net
\end{itemize}

两种方法均与M3单模型(Fusion-Net)进行比较。

\section{诊断准确性指标}

\begin{table}[H]
\centering
\caption{\textbf{表1:内部测试集和前瞻性测试集中不同方法评估肝纤维化的诊断准确性指标}}
\label{tab:diagnostic_accuracy}
\footnotesize
\renewcommand{\arraystretch}{1.3}
\begin{tabular}{p{4.2cm}cccccc}
\toprule
场景和纤维化检测 & 阈值 & 准确率 & 敏感性 & 特异性 & PPV & NPV \\
\midrule
\rowcolor{gray!15}
\multicolumn{7}{l}{\textbf{内部测试集} ($n$ = 51)} \\
\quad Fusion-Net (M3) & 0.09 & 84 (43/51) & 91 (20/22) & 79 (23/29) & 77 (20/26) & 92 (23/25) \\
\rowcolor{gray!5}
\quad \textbf{Two-step\#1 (M4$\rightarrow$M3)} & 0.07; 0.10 & \textbf{86 (44/51)} & \textbf{91 (20/22)} & 83 (24/29) & 80 (20/25) & \textbf{92 (24/26)} \\
\quad Two-step\#2 (M5$\rightarrow$M3) & 0.30; 0.09 & 86 (44/51) & 86 (19/22) & 86 (25/29) & 83 (19/23) & 89 (25/28) \\
\midrule
\rowcolor{gray!15}
\multicolumn{7}{l}{\textbf{前瞻性测试集} ($n$ = 100)} \\
\quad Fusion-Net (M3) & 0.05 & 85 (85/100) & 95 (37/39) & 79 (48/61) & 74 (37/50) & 96 (48/50) \\
\rowcolor{gray!5}
\quad \textbf{Two-step\#1 (M4$\rightarrow$M3)} & 0.19; 0.04 & \textbf{86 (86/100)} & \textbf{92 (36/39)} & \textbf{82 (50/61)} & \textbf{77 (36/47)} & \textbf{94 (50/53)} \\
\quad Two-step\#2 (M5$\rightarrow$M3) & 0.40; 0.04 & 85 (85/100) & 87 (34/39) & 84 (51/61) & 77 (34/44) & 91 (51/56) \\
\bottomrule
\end{tabular}

\vspace{2mm}
\begin{minipage}{\textwidth}
\footnotesize
\textit{注:}除非另有说明,数据为百分比,括号内为患者数量。M3 (Fusion-Net)是完整多模态融合模型,M4 (Clinical-A)是临床参数模型,M5 (Clinical-Net)是基础临床深度学习模型。\textbf{加粗}表示最优两步配置。
\end{minipage}
\end{table}

\section{转诊影响分析}

\begin{table}[H]
\centering
\caption{\textbf{表2:内部测试集和前瞻性测试集中序贯算法的转诊影响}}
\label{tab:referral_impact}
\footnotesize
\renewcommand{\arraystretch}{1.3}
\begin{tabular}{lcccc}
\toprule
 & Step 1 & Step 1 & 避免转诊 & \\
算法 & 阴性排除 & 阳性进入 & 比例 (\%) & $\Delta$NPV \\
\midrule
\rowcolor{gray!15}
\multicolumn{5}{l}{\textbf{内部测试集} ($n$ = 51)} \\
\textbf{M4$\rightarrow$M3} & 6 & 45 & 12\% & +0.3\% \\
\rowcolor{gray!5}
M5$\rightarrow$M3 & 13 & 38 & 25\% & $-$2.7\% \\
\midrule
\rowcolor{gray!15}
\multicolumn{5}{l}{\textbf{前瞻性测试集} ($n$ = 100)} \\
\textbf{M4$\rightarrow$M3} & 25 & 75 & 25\% & $-$1.7\% \\
\rowcolor{gray!5}
M5$\rightarrow$M3 & 35 & 65 & 35\% & $-$4.9\% \\
\bottomrule
\end{tabular}

\vspace{2mm}
\begin{minipage}{\textwidth}
\footnotesize
\textit{注:}$\Delta$NPV表示与M3单模型相比的NPV变化。\textbf{M4$\rightarrow$M3}在保持更高NPV的同时实现有效转诊节省。
\end{minipage}
\end{table}

\section{序贯筛查流程}

\begin{figure}[H]
\centering
\begin{minipage}{0.48\textwidth}
\centering
\includegraphics[width=\textwidth]{figures/sankey_Internal_M4_to_M3.png}
\end{minipage}
\hfill
\begin{minipage}{0.48\textwidth}
\centering
\includegraphics[width=\textwidth]{figures/sankey_Internal_M5_to_M3.png}
\end{minipage}
\caption{内部测试集的序贯筛查算法流程图。左:\textbf{M4$\rightarrow$M3}(排除6人,12\%);右:M5$\rightarrow$M3(排除13人,25\%)}
\label{fig:sankey_internal}
\end{figure}

\begin{figure}[H]
\centering
\begin{minipage}{0.48\textwidth}
\centering
\includegraphics[width=\textwidth]{figures/sankey_Prospective_M4_to_M3.png}
\end{minipage}
\hfill
\begin{minipage}{0.48\textwidth}
\centering
\includegraphics[width=\textwidth]{figures/sankey_Prospective_M5_to_M3.png}
\end{minipage}
\caption{前瞻性测试集的序贯筛查算法流程图。左:\textbf{M4$\rightarrow$M3}(排除25人,25\%);右:M5$\rightarrow$M3(排除35人,35\%)}
\label{fig:sankey_prospective}
\end{figure}

\section{预测值与患病率关系}

\begin{figure}[H]
\centering
\includegraphics[width=\textwidth]{figures/ppv_npv_curves.png}
\caption{三种方法(M3、M4$\rightarrow$M3、M5$\rightarrow$M3)的PPV和NPV随患病率变化曲线。实线为PPV,虚线为NPV。}
\label{fig:ppv_npv}
\end{figure}

\section{主要发现}

\subsection{M4$\rightarrow$M3为最优配置}

\textbf{内部测试集:}
\begin{itemize}
\item 准确率:86\%(优于M3的84\%)$\uparrow$
\item 敏感性:91\%(与M3持平)
\item NPV:92\%(与M3持平)
\item 避免转诊:12\%
\end{itemize}

\textbf{前瞻性测试集:}
\begin{itemize}
\item 准确率:86\%(优于M3的85\%)$\uparrow$
\item 敏感性:92\%
\item NPV:94\%
\item 避免转诊:25\%
\end{itemize}

\subsection{M4$\rightarrow$M3优于M5$\rightarrow$M3}

在两个测试集中,M4$\rightarrow$M3均优于M5$\rightarrow$M3:
\begin{itemize}
\item \textbf{内部测试集}:敏感性91\% vs 86\%,NPV 92\% vs 89\%
\item \textbf{前瞻性测试集}:准确率86\% vs 85\%,敏感性92\% vs 87\%,NPV 94\% vs 91\%
\end{itemize}

M5$\rightarrow$M3虽然避免转诊率更高(内部25\%,前瞻35\%),但关键指标(敏感性、NPV)较低。

\section{结论}

\begin{enumerate}
\item \textbf{M4$\rightarrow$M3是推荐的最优配置}:在保持高诊断准确性的同时,实现有效的资源节省。

\item \textbf{NPV保持良好}:M4$\rightarrow$M3的NPV下降仅2-3\%,临床可接受。

\item \textbf{权衡考虑}:若资源极度有限,可考虑M5$\rightarrow$M3以获得更高的转诊节省率,但需接受更高的漏诊风险。
\end{enumerate}

\end{document}
