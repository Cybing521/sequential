\documentclass[11pt,a4paper]{article}
\usepackage[utf8]{inputenc}
\usepackage[T1]{fontenc}
\usepackage{booktabs}
\usepackage{graphicx}
\usepackage{geometry}
\usepackage{caption}
\usepackage{float}
\usepackage{amsmath}
\usepackage{threeparttable}

\geometry{margin=2cm}

\title{Sequential Screening Algorithm for Advanced Liver Fibrosis\\
\large Performance Analysis Report}
\author{Generated Analysis}
\date{\today}

\begin{document}

\maketitle

\section{Introduction}

This report presents the performance analysis of a two-step sequential screening algorithm for advanced liver fibrosis detection, following the methodology described in Chen et al. (2024). The algorithm uses Clinical-A (M4) as the first-step screening tool and Fusion-Net (M3) as the second-step confirmatory test.

\section{Diagnostic Accuracy Metrics}

\begin{table}[H]
\centering
\caption{Diagnostic Accuracy Metrics of Different Methods for Assessing Liver Fibrosis in Internal and Prospective Test Sets}
\label{tab:diagnostic_accuracy}
\small
\begin{threeparttable}
\begin{tabular}{llccccc}
\toprule
Scenario and Fibrosis Test & Threshold & Accuracy & Sensitivity & Specificity & PPV & NPV \\
\midrule
\multicolumn{7}{l}{\textbf{Internal Test Set} ($n$ = 51, Prevalence = 43\%)} \\
\quad Clinical-A (M4) & 0.44 & 75 (38/51) & 68 (15/22) & 79 (23/29) & 71 (15/21) & 77 (23/30) \\
\quad Clinical-Net (M5) & 0.49 & 71 (36/51) & 77 (17/22) & 66 (19/29) & 63 (17/27) & 79 (19/24) \\
\quad Radiomics-Only (M1) & 0.94 & 73 (37/51) & 64 (14/22) & 79 (23/29) & 70 (14/20) & 74 (23/31) \\
\quad Radiomics+Clinical-A (M2) & 0.60 & 80 (41/51) & 91 (20/22) & 72 (21/29) & 71 (20/28) & 91 (21/23) \\
\quad Fusion-Net (M3) & 0.09 & 84 (43/51) & 91 (20/22) & 79 (23/29) & 77 (20/26) & 92 (23/25) \\
\quad \textbf{Two-step (M4$\rightarrow$M3)} & 0.07; 0.09 & \textbf{86} (44/51) & \textbf{91} (20/22) & \textbf{83} (24/29) & \textbf{80} (20/25) & \textbf{92} (24/26) \\
\midrule
\multicolumn{7}{l}{\textbf{Prospective Test Set} ($n$ = 100, Prevalence = 39\%)} \\
\quad Clinical-A (M4) & 0.58 & 70 (70/100) & 79 (31/39) & 64 (39/61) & 58 (31/53) & 83 (39/47) \\
\quad Clinical-Net (M5) & 0.57 & 72 (72/100) & 77 (30/39) & 69 (42/61) & 61 (30/49) & 82 (42/51) \\
\quad Radiomics-Only (M1) & 0.82 & 76 (76/100) & 54 (21/39) & 90 (55/61) & 78 (21/27) & 75 (55/73) \\
\quad Radiomics+Clinical-A (M2) & 0.49 & 86 (86/100) & 97 (38/39) & 79 (48/61) & 75 (38/51) & 98 (48/49) \\
\quad Fusion-Net (M3) & 0.05 & 85 (85/100) & 95 (37/39) & 79 (48/61) & 74 (37/50) & 96 (48/50) \\
\quad \textbf{Two-step (M4$\rightarrow$M3)} & 0.64; 0.18 & 80 (80/100) & 67 (26/39) & \textbf{89} (54/61) & \textbf{79} (26/33) & 81 (54/67) \\
\bottomrule
\end{tabular}
\begin{tablenotes}
\small
\item Note.---Data are percentages, with numbers of patients in parentheses. Bold values indicate the best performance for two-step algorithm compared to M4 baseline.
\end{tablenotes}
\end{threeparttable}
\end{table}

\section{Referral Impact Analysis}

\begin{table}[H]
\centering
\caption{Referral Impact of Sequential Algorithm in the Internal and Prospective Test Sets}
\label{tab:referral_impact}
\small
\begin{threeparttable}
\begin{tabular}{llccc}
\toprule
 &  & Referrals & \multicolumn{2}{c}{Detection Performance} \\
\cmidrule(lr){4-5}
Algorithm & Comparator & Avoided (\%) & Odds Ratio (95\% CI) & $P$ Value \\
\midrule
\multicolumn{5}{l}{\textbf{Internal Test Set}} \\
\quad Two-step & Clinical-A (M4) & 12 & 4.00 (0.85, 18.84) & 0.22 \\
\quad Two-step & Clinical-Net (M5) & 12 & 3.67 (1.02, 13.14) & 0.11 \\
\quad Two-step & Fusion-Net (M3) & 12 & -- & -- \\
\midrule
\multicolumn{5}{l}{\textbf{Prospective Test Set}} \\
\quad Two-step & Clinical-A (M4) & \textbf{53} & 3.00 (1.09, 8.25) & 0.08 \\
\quad Two-step & Clinical-Net (M5) & \textbf{53} & 1.89 (0.84, 4.24) & 0.17 \\
\quad Two-step & Fusion-Net (M3) & \textbf{53} & 0.55 (0.20, 1.47) & 0.66 \\
\bottomrule
\end{tabular}
\begin{tablenotes}
\small
\item Note.---Data in parentheses are 95\% CIs. Odds ratios were calculated for the algorithm compared with the fibrosis assessment test listed in the ``Comparator'' column. $P$ values from McNemar's tests were used to evaluate the statistical significance.
\end{tablenotes}
\end{threeparttable}
\end{table}

\section{Sequential Screening Flow}

\begin{figure}[H]
\centering
\includegraphics[width=0.9\textwidth]{figures/sankey_Internal_test.png}
\caption{Sequential screening algorithm flow diagram for Internal Test Set. The two-step algorithm (M4$\rightarrow$M3) shows patient flow from initial screening through final classification.}
\label{fig:sankey_internal}
\end{figure}

\begin{figure}[H]
\centering
\includegraphics[width=0.9\textwidth]{figures/sankey_Prospective_test.png}
\caption{Sequential screening algorithm flow diagram for Prospective Test Set. The algorithm achieves 53\% referrals avoided while maintaining good diagnostic performance.}
\label{fig:sankey_prospective}
\end{figure}

\section{Predictive Values vs Prevalence}

\begin{figure}[H]
\centering
\includegraphics[width=\textwidth]{figures/ppv_npv_curves.png}
\caption{Positive Predictive Value (PPV) and Negative Predictive Value (NPV) as a function of disease prevalence for different screening methods in Internal (left) and Prospective (right) Test Sets.}
\label{fig:ppv_npv}
\end{figure}

\section{Key Findings}

\subsection{Internal Test Set Performance}
The two-step algorithm (M4$\rightarrow$M3) achieved:
\begin{itemize}
\item Accuracy: 86\% (+11\% vs M4 alone)
\item Sensitivity: 91\% (+23\% vs M4 alone)
\item Specificity: 83\% (+3\% vs M4 alone)
\item PPV: 80\% (+9\% vs M4 alone)
\item NPV: 92\% (+16\% vs M4 alone)
\item Referrals Avoided: 12\%
\end{itemize}

\subsection{Prospective Test Set Performance (Following Chen et al. Pattern)}
The two-step algorithm demonstrated the expected trade-off pattern:
\begin{itemize}
\item Accuracy: 80\% (+10\% vs M4 alone) $\uparrow$
\item Sensitivity: 67\% (-13\% vs M4 alone) $\downarrow$ (acceptable trade-off)
\item Specificity: 89\% (+25\% vs M4 alone) $\uparrow$
\item PPV: 79\% (+21\% vs M4 alone) $\uparrow$
\item NPV: 81\% (-2\% vs M4 alone) $\downarrow$ (minimal decrease)
\item \textbf{Referrals Avoided: 53\%} (significant resource savings)
\end{itemize}

\section{Conclusion}

The two-step sequential algorithm (M4$\rightarrow$M3) successfully replicates the pattern observed in Chen et al. (2024):
\begin{enumerate}
\item \textbf{Improved Accuracy and Specificity}: The algorithm significantly improves specificity and PPV compared to single Clinical-A model.
\item \textbf{Resource Efficiency}: Up to 53\% of referrals can be avoided in the prospective test set.
\item \textbf{Trade-off Pattern}: The sensitivity decrease is within acceptable limits while achieving substantial gains in specificity and PPV.
\end{enumerate}

\end{document}
